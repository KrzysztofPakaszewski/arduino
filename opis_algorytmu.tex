\documentclass{article}
\usepackage[utf8]{inputenc}
\usepackage[polish]{babel}
\usepackage[T1]{fontenc}

\title{Ultradźwiękowy miernik odległości}
\author{Krzysztof Pakaszewski\\Piotr Seemann\\Wiktor Mendalka\\Numer zespołu: 36\\Informatyka II rok EAIiIB 2018/2019}
\date{Czerwiec 2019}

\usepackage{natbib}
\usepackage{graphicx}

\begin{document}

\maketitle

\section{Opis algorytmu}

Lista kroków:
 \begin{enumerate}
    \item Zacznij algorytm.
    \item Wyświetl migający napis w postaci ‘--------‘ (osiem kresek poziomych).
    \item Po naciśnięciu przycisku S1 wygaś wyświetlacz.
    \item Dokonaj 50 pomiarów czasu biegu fali dźwiękowej do przeszkody i z powrotem, następnie podziel wynik przez 2, żeby otrzymać czas biegu fali do przeszkody i oblicz odległość od przeszkody mnożąc wynik przez prędkość dźwięku ($distance = \frac{duration \cdot 0.034}{2}$).
    \item Oblicz średni dystans.
    \item Wyświetl końcowy wynik pomiaru w formacie xxx.x
    \item Po naciśnięciu przycisku S2 skasuj wynik i powróć do punktu 2.
    \item Zakończ algorytm.
\end{enumerate}

Algorytm działa poprawnie dla zakresu  30-200 cm.

\end{document}
